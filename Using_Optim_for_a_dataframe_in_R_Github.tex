\documentclass[]{article}
\usepackage{lmodern}
\usepackage{amssymb,amsmath}
\usepackage{ifxetex,ifluatex}
\usepackage{fixltx2e} % provides \textsubscript
\ifnum 0\ifxetex 1\fi\ifluatex 1\fi=0 % if pdftex
  \usepackage[T1]{fontenc}
  \usepackage[utf8]{inputenc}
\else % if luatex or xelatex
  \ifxetex
    \usepackage{mathspec}
  \else
    \usepackage{fontspec}
  \fi
  \defaultfontfeatures{Ligatures=TeX,Scale=MatchLowercase}
\fi
% use upquote if available, for straight quotes in verbatim environments
\IfFileExists{upquote.sty}{\usepackage{upquote}}{}
% use microtype if available
\IfFileExists{microtype.sty}{%
\usepackage{microtype}
\UseMicrotypeSet[protrusion]{basicmath} % disable protrusion for tt fonts
}{}
\usepackage[margin=1in]{geometry}
\usepackage{hyperref}
\hypersetup{unicode=true,
            pdftitle={Watershed Capstone Project - Price Optimization},
            pdfborder={0 0 0},
            breaklinks=true}
\urlstyle{same}  % don't use monospace font for urls
\usepackage{color}
\usepackage{fancyvrb}
\newcommand{\VerbBar}{|}
\newcommand{\VERB}{\Verb[commandchars=\\\{\}]}
\DefineVerbatimEnvironment{Highlighting}{Verbatim}{commandchars=\\\{\}}
% Add ',fontsize=\small' for more characters per line
\usepackage{framed}
\definecolor{shadecolor}{RGB}{248,248,248}
\newenvironment{Shaded}{\begin{snugshade}}{\end{snugshade}}
\newcommand{\AlertTok}[1]{\textcolor[rgb]{0.94,0.16,0.16}{#1}}
\newcommand{\AnnotationTok}[1]{\textcolor[rgb]{0.56,0.35,0.01}{\textbf{\textit{#1}}}}
\newcommand{\AttributeTok}[1]{\textcolor[rgb]{0.77,0.63,0.00}{#1}}
\newcommand{\BaseNTok}[1]{\textcolor[rgb]{0.00,0.00,0.81}{#1}}
\newcommand{\BuiltInTok}[1]{#1}
\newcommand{\CharTok}[1]{\textcolor[rgb]{0.31,0.60,0.02}{#1}}
\newcommand{\CommentTok}[1]{\textcolor[rgb]{0.56,0.35,0.01}{\textit{#1}}}
\newcommand{\CommentVarTok}[1]{\textcolor[rgb]{0.56,0.35,0.01}{\textbf{\textit{#1}}}}
\newcommand{\ConstantTok}[1]{\textcolor[rgb]{0.00,0.00,0.00}{#1}}
\newcommand{\ControlFlowTok}[1]{\textcolor[rgb]{0.13,0.29,0.53}{\textbf{#1}}}
\newcommand{\DataTypeTok}[1]{\textcolor[rgb]{0.13,0.29,0.53}{#1}}
\newcommand{\DecValTok}[1]{\textcolor[rgb]{0.00,0.00,0.81}{#1}}
\newcommand{\DocumentationTok}[1]{\textcolor[rgb]{0.56,0.35,0.01}{\textbf{\textit{#1}}}}
\newcommand{\ErrorTok}[1]{\textcolor[rgb]{0.64,0.00,0.00}{\textbf{#1}}}
\newcommand{\ExtensionTok}[1]{#1}
\newcommand{\FloatTok}[1]{\textcolor[rgb]{0.00,0.00,0.81}{#1}}
\newcommand{\FunctionTok}[1]{\textcolor[rgb]{0.00,0.00,0.00}{#1}}
\newcommand{\ImportTok}[1]{#1}
\newcommand{\InformationTok}[1]{\textcolor[rgb]{0.56,0.35,0.01}{\textbf{\textit{#1}}}}
\newcommand{\KeywordTok}[1]{\textcolor[rgb]{0.13,0.29,0.53}{\textbf{#1}}}
\newcommand{\NormalTok}[1]{#1}
\newcommand{\OperatorTok}[1]{\textcolor[rgb]{0.81,0.36,0.00}{\textbf{#1}}}
\newcommand{\OtherTok}[1]{\textcolor[rgb]{0.56,0.35,0.01}{#1}}
\newcommand{\PreprocessorTok}[1]{\textcolor[rgb]{0.56,0.35,0.01}{\textit{#1}}}
\newcommand{\RegionMarkerTok}[1]{#1}
\newcommand{\SpecialCharTok}[1]{\textcolor[rgb]{0.00,0.00,0.00}{#1}}
\newcommand{\SpecialStringTok}[1]{\textcolor[rgb]{0.31,0.60,0.02}{#1}}
\newcommand{\StringTok}[1]{\textcolor[rgb]{0.31,0.60,0.02}{#1}}
\newcommand{\VariableTok}[1]{\textcolor[rgb]{0.00,0.00,0.00}{#1}}
\newcommand{\VerbatimStringTok}[1]{\textcolor[rgb]{0.31,0.60,0.02}{#1}}
\newcommand{\WarningTok}[1]{\textcolor[rgb]{0.56,0.35,0.01}{\textbf{\textit{#1}}}}
\usepackage{graphicx,grffile}
\makeatletter
\def\maxwidth{\ifdim\Gin@nat@width>\linewidth\linewidth\else\Gin@nat@width\fi}
\def\maxheight{\ifdim\Gin@nat@height>\textheight\textheight\else\Gin@nat@height\fi}
\makeatother
% Scale images if necessary, so that they will not overflow the page
% margins by default, and it is still possible to overwrite the defaults
% using explicit options in \includegraphics[width, height, ...]{}
\setkeys{Gin}{width=\maxwidth,height=\maxheight,keepaspectratio}
\IfFileExists{parskip.sty}{%
\usepackage{parskip}
}{% else
\setlength{\parindent}{0pt}
\setlength{\parskip}{6pt plus 2pt minus 1pt}
}
\setlength{\emergencystretch}{3em}  % prevent overfull lines
\providecommand{\tightlist}{%
  \setlength{\itemsep}{0pt}\setlength{\parskip}{0pt}}
\setcounter{secnumdepth}{0}
% Redefines (sub)paragraphs to behave more like sections
\ifx\paragraph\undefined\else
\let\oldparagraph\paragraph
\renewcommand{\paragraph}[1]{\oldparagraph{#1}\mbox{}}
\fi
\ifx\subparagraph\undefined\else
\let\oldsubparagraph\subparagraph
\renewcommand{\subparagraph}[1]{\oldsubparagraph{#1}\mbox{}}
\fi

%%% Use protect on footnotes to avoid problems with footnotes in titles
\let\rmarkdownfootnote\footnote%
\def\footnote{\protect\rmarkdownfootnote}

%%% Change title format to be more compact
\usepackage{titling}

% Create subtitle command for use in maketitle
\providecommand{\subtitle}[1]{
  \posttitle{
    \begin{center}\large#1\end{center}
    }
}

\setlength{\droptitle}{-2em}

  \title{Watershed Capstone Project - Price Optimization}
    \pretitle{\vspace{\droptitle}\centering\huge}
  \posttitle{\par}
    \author{}
    \preauthor{}\postauthor{}
    \date{}
    \predate{}\postdate{}
  

\begin{document}
\maketitle

This is the code that I use to run predictive model for forecasted
occupancy rate as a function of nightly rent price. I also use optim in
order to find the short-term rent price that could optimize revenue for
each single property, just like we use Solver in Excel.

\begin{Shaded}
\begin{Highlighting}[]
\NormalTok{df <-}\StringTok{ }\KeywordTok{read.csv}\NormalTok{(}\DataTypeTok{file =} \StringTok{"database.csv"}\NormalTok{, }\DataTypeTok{header=}\OtherTok{TRUE}\NormalTok{)}
\KeywordTok{head}\NormalTok{(df)}
\end{Highlighting}
\end{Shaded}

\begin{verbatim}
##   location property_type ws_property_id current_monthly_rent
## 1    L9531            R6             W1                 1060
## 2    L9533            R6            W10                 1200
## 3    L1944            R2           W100                 3300
## 4   L15257            R2           W101                 1400
## 5   L15257            R6           W102                 2000
## 6   L15257           R10           W103                 1600
##   percentile_10th_price percentile_90th_price sample_nightly_rent_price
## 1                   114                   153                       148
## 2                   111                   149                       133
## 3                   108                   610                       372
## 4                   178                   533                       302
## 5                   221                   617                       429
## 6                   202                   646                       380
##   occupancy_rate          city state zipcode apt_house num_bedrooms
## 1           0.16   Chapel Hill    NC   27514 apartment            2
## 2           0.35   Chapel Hill    NC   27517 apartment            2
## 3           0.40 San Francisco    CA   94129 apartment            1
## 4           0.36        Austin    TX   78702 apartment            1
## 5           0.41        Austin    TX   78702 apartment            2
## 6           0.41        Austin    TX   78702     house            1
##   kitchen shared
## 1       Y      N
## 2       Y      N
## 3       Y      N
## 4       Y      N
## 5       Y      N
## 6       Y      N
\end{verbatim}

Our purpose is to predict occupancy rate using current variables, so
let's run a simple prediction model with independent variable being
sample\_nightly\_rent\_price.

\begin{Shaded}
\begin{Highlighting}[]
\NormalTok{model<-}\KeywordTok{lm}\NormalTok{(}\DataTypeTok{data=}\NormalTok{df, occupancy_rate }\OperatorTok{~}\StringTok{ }\NormalTok{sample_nightly_rent_price)}
\KeywordTok{summary}\NormalTok{(model)}
\end{Highlighting}
\end{Shaded}

\begin{verbatim}
## 
## Call:
## lm(formula = occupancy_rate ~ sample_nightly_rent_price, data = df)
## 
## Residuals:
##      Min       1Q   Median       3Q      Max 
## -0.40692 -0.10288 -0.00011  0.09632  0.46844 
## 
## Coefficients:
##                             Estimate Std. Error t value Pr(>|t|)    
## (Intercept)                5.086e-01  2.121e-02  23.976  < 2e-16 ***
## sample_nightly_rent_price -1.524e-04  5.373e-05  -2.837  0.00494 ** 
## ---
## Signif. codes:  0 '***' 0.001 '**' 0.01 '*' 0.05 '.' 0.1 ' ' 1
## 
## Residual standard error: 0.1615 on 242 degrees of freedom
## Multiple R-squared:  0.03219,    Adjusted R-squared:  0.02819 
## F-statistic: 8.048 on 1 and 242 DF,  p-value: 0.004941
\end{verbatim}

This is obviously a very low-accurate model with R-squared being only a
mere 3\%. Intuitively, each cities and each regions will have different
range of rent prices. Renting a small bedromm at \$200/night is
obviously too expensive in a small town at Montana; however, it could be
considered as acceptable in a metropolis such as New York.

Thus, to capture this effect, we should incorporate the different in
living expenses at different regions into our model, which could be done
by reflecting each rent price as a percentile vs.~prices of other
properties in the same region. If \$200 is at 90 percentile in Montana
and at 50 percentile in New York, it would be better for the regression
model to predict.

Thankfully, the given dataset allows us to do this practice.

\begin{Shaded}
\begin{Highlighting}[]
\NormalTok{col<-}\KeywordTok{c}\NormalTok{(}\StringTok{"percentile_90th_price"}\NormalTok{,}\StringTok{"percentile_10th_price"}\NormalTok{,}\StringTok{"sample_nightly_rent_price"}\NormalTok{)}
\NormalTok{dt<-df[col]}

\NormalTok{vs10 <-}\StringTok{ }\ControlFlowTok{function}\NormalTok{ (x) \{}\ControlFlowTok{if}\NormalTok{ (x[}\DecValTok{3}\NormalTok{]}\OperatorTok{>}\NormalTok{x[}\DecValTok{2}\NormalTok{]) \{}\DecValTok{1}\NormalTok{\} }\ControlFlowTok{else}\NormalTok{ \{}\DecValTok{0}\NormalTok{\}\}}
\NormalTok{vs90 <-}\StringTok{ }\ControlFlowTok{function}\NormalTok{ (x) \{}\ControlFlowTok{if}\NormalTok{ (x[}\DecValTok{3}\NormalTok{]}\OperatorTok{<}\NormalTok{x[}\DecValTok{1}\NormalTok{]) \{}\DecValTok{1}\NormalTok{\} }\ControlFlowTok{else}\NormalTok{ \{}\DecValTok{0}\NormalTok{\}\}}


\NormalTok{dt}\OperatorTok{$}\NormalTok{vs10 <-}\StringTok{ }\KeywordTok{apply}\NormalTok{(dt,}\DecValTok{1}\NormalTok{,vs10) }\CommentTok{#check if sample rent price is higher than percentile 10}
\NormalTok{dt}\OperatorTok{$}\NormalTok{vs90 <-}\StringTok{ }\KeywordTok{apply}\NormalTok{(dt,}\DecValTok{1}\NormalTok{,vs90) }\CommentTok{#check if sample rent price is lower than percentile 90}

\KeywordTok{c}\NormalTok{(}\KeywordTok{sum}\NormalTok{(dt}\OperatorTok{$}\NormalTok{vs90),}\KeywordTok{sum}\NormalTok{(dt}\OperatorTok{$}\NormalTok{vs10))}
\end{Highlighting}
\end{Shaded}

\begin{verbatim}
## [1] 243 243
\end{verbatim}

Since the total dataset has 243 instances, this indicates that all the
sample rent price is within (10,90) percentile.

We can then convert the sample nightly rent price into percentile using
below formula:

sample\_percentile\_price of a given property (x) = 0.1+0.8*(range
between x and 10th percentile )/(range between 90 and 10 percentile)

\begin{Shaded}
\begin{Highlighting}[]
\NormalTok{df}\OperatorTok{$}\NormalTok{percentile_90th_vs_10th =df}\OperatorTok{$}\NormalTok{percentile_90th_price}\OperatorTok{-}\NormalTok{df}\OperatorTok{$}\NormalTok{percentile_10th_price}
\NormalTok{df}\OperatorTok{$}\NormalTok{sample_vs_10th <-}\StringTok{ }\NormalTok{df}\OperatorTok{$}\NormalTok{sample_nightly_rent_price}\OperatorTok{-}\NormalTok{df}\OperatorTok{$}\NormalTok{percentile_10th_price}
\NormalTok{df}\OperatorTok{$}\NormalTok{sample_price_percentile <-}\StringTok{ }\FloatTok{0.1+0.8}\OperatorTok{*}\NormalTok{df}\OperatorTok{$}\NormalTok{sample_vs_10th}\OperatorTok{/}\NormalTok{df}\OperatorTok{$}\NormalTok{percentile_90th_vs_10th}

\KeywordTok{head}\NormalTok{(df)}
\end{Highlighting}
\end{Shaded}

\begin{verbatim}
##   location property_type ws_property_id current_monthly_rent
## 1    L9531            R6             W1                 1060
## 2    L9533            R6            W10                 1200
## 3    L1944            R2           W100                 3300
## 4   L15257            R2           W101                 1400
## 5   L15257            R6           W102                 2000
## 6   L15257           R10           W103                 1600
##   percentile_10th_price percentile_90th_price sample_nightly_rent_price
## 1                   114                   153                       148
## 2                   111                   149                       133
## 3                   108                   610                       372
## 4                   178                   533                       302
## 5                   221                   617                       429
## 6                   202                   646                       380
##   occupancy_rate          city state zipcode apt_house num_bedrooms
## 1           0.16   Chapel Hill    NC   27514 apartment            2
## 2           0.35   Chapel Hill    NC   27517 apartment            2
## 3           0.40 San Francisco    CA   94129 apartment            1
## 4           0.36        Austin    TX   78702 apartment            1
## 5           0.41        Austin    TX   78702 apartment            2
## 6           0.41        Austin    TX   78702     house            1
##   kitchen shared percentile_90th_vs_10th sample_vs_10th
## 1       Y      N                      39             34
## 2       Y      N                      38             22
## 3       Y      N                     502            264
## 4       Y      N                     355            124
## 5       Y      N                     396            208
## 6       Y      N                     444            178
##   sample_price_percentile
## 1               0.7974359
## 2               0.5631579
## 3               0.5207171
## 4               0.3794366
## 5               0.5202020
## 6               0.4207207
\end{verbatim}

Now, let's rerun the regression model

\begin{Shaded}
\begin{Highlighting}[]
\NormalTok{model<-}\KeywordTok{lm}\NormalTok{(}\DataTypeTok{data=}\NormalTok{df, occupancy_rate }\OperatorTok{~}\StringTok{ }\NormalTok{sample_price_percentile)}
\KeywordTok{summary}\NormalTok{(model)}
\end{Highlighting}
\end{Shaded}

\begin{verbatim}
## 
## Call:
## lm(formula = occupancy_rate ~ sample_price_percentile, data = df)
## 
## Residuals:
##      Min       1Q   Median       3Q      Max 
## -0.37493 -0.06930  0.00154  0.07194  0.30655 
## 
## Coefficients:
##                         Estimate Std. Error t value Pr(>|t|)    
## (Intercept)              0.85167    0.02560   33.27   <2e-16 ***
## sample_price_percentile -0.79359    0.04922  -16.12   <2e-16 ***
## ---
## Signif. codes:  0 '***' 0.001 '**' 0.01 '*' 0.05 '.' 0.1 ' ' 1
## 
## Residual standard error: 0.114 on 242 degrees of freedom
## Multiple R-squared:  0.5179, Adjusted R-squared:  0.5159 
## F-statistic:   260 on 1 and 242 DF,  p-value: < 2.2e-16
\end{verbatim}

I also tested inputting number of bedrooms into our model. However, this
variable does not yield statistically significance. Perhaps the reason
is that the rent price already factors the number of bedrooms, thus
incorporating such new feature does not help much to our prediction.

\begin{Shaded}
\begin{Highlighting}[]
\NormalTok{model_}\DecValTok{1}\NormalTok{<-}\KeywordTok{lm}\NormalTok{(}\DataTypeTok{data=}\NormalTok{df, occupancy_rate }\OperatorTok{~}\StringTok{ }\NormalTok{sample_price_percentile }\OperatorTok{+}\StringTok{ }\NormalTok{num_bedrooms)}
\KeywordTok{summary}\NormalTok{(model_}\DecValTok{1}\NormalTok{)}
\end{Highlighting}
\end{Shaded}

\begin{verbatim}
## 
## Call:
## lm(formula = occupancy_rate ~ sample_price_percentile + num_bedrooms, 
##     data = df)
## 
## Residuals:
##      Min       1Q   Median       3Q      Max 
## -0.37028 -0.07003  0.00165  0.07053  0.31110 
## 
## Coefficients:
##                          Estimate Std. Error t value Pr(>|t|)    
## (Intercept)              0.865209   0.033469  25.851   <2e-16 ***
## sample_price_percentile -0.793088   0.049287 -16.091   <2e-16 ***
## num_bedrooms            -0.009193   0.014615  -0.629     0.53    
## ---
## Signif. codes:  0 '***' 0.001 '**' 0.01 '*' 0.05 '.' 0.1 ' ' 1
## 
## Residual standard error: 0.1141 on 241 degrees of freedom
## Multiple R-squared:  0.5187, Adjusted R-squared:  0.5147 
## F-statistic: 129.9 on 2 and 241 DF,  p-value: < 2.2e-16
\end{verbatim}

I also want to plot QQ chart and residual plot to test if our univariate
model follows the assumption of linear regression.

\begin{Shaded}
\begin{Highlighting}[]
\NormalTok{model.res<-}\KeywordTok{resid}\NormalTok{(model);}

\KeywordTok{plot}\NormalTok{(df}\OperatorTok{$}\NormalTok{sample_price_percentile, model.res, }
        \DataTypeTok{ylab=}\StringTok{"Residuals"}\NormalTok{, }\DataTypeTok{xlab=}\StringTok{"Sample Price Percentile"}\NormalTok{) }
\KeywordTok{abline}\NormalTok{(}\DecValTok{0}\NormalTok{,}\DecValTok{0}\NormalTok{);}
\end{Highlighting}
\end{Shaded}

\includegraphics{Using_Optim_for_a_dataframe_in_R_Github_files/figure-latex/unnamed-chunk-7-1.pdf}

\begin{Shaded}
\begin{Highlighting}[]
\KeywordTok{qqnorm}\NormalTok{(model.res, }\DataTypeTok{pch =} \DecValTok{1}\NormalTok{, }\DataTypeTok{frame =} \OtherTok{FALSE}\NormalTok{)}
\KeywordTok{qqline}\NormalTok{(model.res, }\DataTypeTok{col =} \StringTok{"steelblue"}\NormalTok{, }\DataTypeTok{lwd =} \DecValTok{2}\NormalTok{)}
\end{Highlighting}
\end{Shaded}

\includegraphics{Using_Optim_for_a_dataframe_in_R_Github_files/figure-latex/unnamed-chunk-7-2.pdf}

Though the R-squared is only moderate, at 55\%, considering current
available dataset and other more important tasks in this analysis, I
decide to go with this univariate model.

Our next step is to find optimal rent price for each property that can
maximize our revenue. Higher rent price could lead to higher income for
us, but it would also lower occupancy rate.

This requirement can easily be done in Excel Solver; nonetheless,
running optimization for each property (total 243 instances) is an
exhausting practice. Consequently, we need to use optim function in R.

\begin{Shaded}
\begin{Highlighting}[]
\CommentTok{# Firsly, we need to create a function to calculate revenue, based on predicted occupancy rate and nightly_rent_price}

\NormalTok{revenue <-}\StringTok{ }\ControlFlowTok{function}\NormalTok{(data,par)\{}
\NormalTok{                            par_vs_10th <-}\StringTok{ }\NormalTok{par}\OperatorTok{-}\NormalTok{data}\OperatorTok{$}\NormalTok{percentile_10th_price}
\NormalTok{                            normalized_price <-}\FloatTok{0.1+0.8}\OperatorTok{*}\NormalTok{par_vs_10th}\OperatorTok{/}\NormalTok{data}\OperatorTok{$}\NormalTok{percentile_90th_vs_10th}
\NormalTok{                            fcst_occupancy <-}\KeywordTok{coef}\NormalTok{(model)[}\StringTok{'(Intercept)'}\NormalTok{]}\OperatorTok{+}\KeywordTok{coef}\NormalTok{(model)[}\StringTok{'sample_price_percentile'}\NormalTok{]}\OperatorTok{*}\NormalTok{normalized_price}
\NormalTok{                            fcst_st_revenue <-fcst_occupancy}\OperatorTok{*}\DecValTok{365}\OperatorTok{*}\NormalTok{par}
\NormalTok{                            fcst_st_revenue}
\NormalTok{\}}

\CommentTok{# Then create an optimization function and run for each row in df.}

\ControlFlowTok{for}\NormalTok{ (i }\ControlFlowTok{in} \DecValTok{1}\OperatorTok{:}\KeywordTok{nrow}\NormalTok{(df))   \{df[i,}\StringTok{'optimized_price'}\NormalTok{] <-}
\StringTok{                                                }\KeywordTok{optim}\NormalTok{(}\DecValTok{122}\NormalTok{,revenue,}\DataTypeTok{data=}\NormalTok{df[i,],}\DataTypeTok{method=}\StringTok{"L-BFGS-B"}\NormalTok{,      }\DataTypeTok{control=}\KeywordTok{list}\NormalTok{(}\DataTypeTok{fnscale=}\OperatorTok{-}\DecValTok{1}\NormalTok{),}\DataTypeTok{lower=}\NormalTok{df[i,}\StringTok{'percentile_10th_price'}\NormalTok{]) \}}
\CommentTok{#122 is an initialized value}
\CommentTok{# We need to input lower parameters to ensure that the optimal rent price does not fall lower than 10th percentile}
\end{Highlighting}
\end{Shaded}

\begin{Shaded}
\begin{Highlighting}[]
\KeywordTok{head}\NormalTok{(df)}
\end{Highlighting}
\end{Shaded}

\begin{verbatim}
##   location property_type ws_property_id current_monthly_rent
## 1    L9531            R6             W1                 1060
## 2    L9533            R6            W10                 1200
## 3    L1944            R2           W100                 3300
## 4   L15257            R2           W101                 1400
## 5   L15257            R6           W102                 2000
## 6   L15257           R10           W103                 1600
##   percentile_10th_price percentile_90th_price sample_nightly_rent_price
## 1                   114                   153                       148
## 2                   111                   149                       133
## 3                   108                   610                       372
## 4                   178                   533                       302
## 5                   221                   617                       429
## 6                   202                   646                       380
##   occupancy_rate          city state zipcode apt_house num_bedrooms
## 1           0.16   Chapel Hill    NC   27514 apartment            2
## 2           0.35   Chapel Hill    NC   27517 apartment            2
## 3           0.40 San Francisco    CA   94129 apartment            1
## 4           0.36        Austin    TX   78702 apartment            1
## 5           0.41        Austin    TX   78702 apartment            2
## 6           0.41        Austin    TX   78702     house            1
##   kitchen shared percentile_90th_vs_10th sample_vs_10th
## 1       Y      N                      39             34
## 2       Y      N                      38             22
## 3       Y      N                     502            264
## 4       Y      N                     355            124
## 5       Y      N                     396            208
## 6       Y      N                     444            178
##   sample_price_percentile optimized_price
## 1               0.7974359        114.0000
## 2               0.5631579        111.0000
## 3               0.5207171        359.3366
## 4               0.3794366        304.9253
## 5               0.5202020        351.3632
## 6               0.4207207        371.0587
\end{verbatim}

We then need to convert the optimized price in dollar values into
percentiles as well. Then use the newly calculated variable to predict
the occupancy rate.

\begin{Shaded}
\begin{Highlighting}[]
\NormalTok{df}\OperatorTok{$}\NormalTok{normalized_optimized_price<-}\FloatTok{0.1+0.8}\OperatorTok{*}\NormalTok{(df}\OperatorTok{$}\NormalTok{optimized_price}\OperatorTok{-}\NormalTok{df}\OperatorTok{$}\NormalTok{percentile_10th_price)}\OperatorTok{/}\NormalTok{(df}\OperatorTok{$}\NormalTok{percentile_90th_vs_10th)}

\NormalTok{new <-}\KeywordTok{data.frame}\NormalTok{(}\DataTypeTok{sample_price_percentile=}\NormalTok{df}\OperatorTok{$}\NormalTok{normalized_optimized_price)}
\NormalTok{df}\OperatorTok{$}\NormalTok{forecast_occupancy <-}\StringTok{ }\KeywordTok{predict.lm}\NormalTok{(model,}\DataTypeTok{newdata=}\NormalTok{new)}

\KeywordTok{head}\NormalTok{(df)}
\end{Highlighting}
\end{Shaded}

\begin{verbatim}
##   location property_type ws_property_id current_monthly_rent
## 1    L9531            R6             W1                 1060
## 2    L9533            R6            W10                 1200
## 3    L1944            R2           W100                 3300
## 4   L15257            R2           W101                 1400
## 5   L15257            R6           W102                 2000
## 6   L15257           R10           W103                 1600
##   percentile_10th_price percentile_90th_price sample_nightly_rent_price
## 1                   114                   153                       148
## 2                   111                   149                       133
## 3                   108                   610                       372
## 4                   178                   533                       302
## 5                   221                   617                       429
## 6                   202                   646                       380
##   occupancy_rate          city state zipcode apt_house num_bedrooms
## 1           0.16   Chapel Hill    NC   27514 apartment            2
## 2           0.35   Chapel Hill    NC   27517 apartment            2
## 3           0.40 San Francisco    CA   94129 apartment            1
## 4           0.36        Austin    TX   78702 apartment            1
## 5           0.41        Austin    TX   78702 apartment            2
## 6           0.41        Austin    TX   78702     house            1
##   kitchen shared percentile_90th_vs_10th sample_vs_10th
## 1       Y      N                      39             34
## 2       Y      N                      38             22
## 3       Y      N                     502            264
## 4       Y      N                     355            124
## 5       Y      N                     396            208
## 6       Y      N                     444            178
##   sample_price_percentile optimized_price normalized_optimized_price
## 1               0.7974359        114.0000                  0.1000000
## 2               0.5631579        111.0000                  0.1000000
## 3               0.5207171        359.3366                  0.5005365
## 4               0.3794366        304.9253                  0.3860289
## 5               0.5202020        351.3632                  0.3633599
## 6               0.4207207        371.0587                  0.4046103
##   forecast_occupancy
## 1          0.7723132
## 2          0.7723132
## 3          0.4544499
## 4          0.5453224
## 5          0.5633124
## 6          0.5305763
\end{verbatim}


\end{document}
